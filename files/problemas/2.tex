\documentclass[11pt]{article}
\usepackage[utf8]{inputenc}
\usepackage[spanish]{babel}
\usepackage{amsmath}
\usepackage{amsfonts}
\usepackage{amssymb}
\usepackage{graphicx}
\usepackage[left=2cm,top=2.5cm,right=2.5cm,bottom=1.5cm]{geometry}
\usepackage[version=3]{mhchem}
\usepackage{chemmacros}
\usepackage{chemformula}
\usepackage{chemfig}
\usepackage[load-headings=true]{exsheets}
\usepackage{enumitem}
\usepackage{marginnote}
\SetupExSheets{headings=margin-nr}
\SetupExSheets[points]{name=\itshape\scriptsize P,number-format=\itshape\scriptsize}
\usepackage{fancybox}
%\newcommand\parens[1]{\ovalbox{#1}}
%\SetupExSheets{points/format=\parens}
\usepackage{marginnote}
% define a new command \subpoints that writes the points in a
% \marginnote; we defined it with a starred variant that only
% will write but not add the points
\makeatletter
\newcommand*\subpoints{\@ifstar\subpoints@star\subpoints@nostar}
\newcommand*\subpoints@star[1]{\marginnote{\points{#1}}}
\newcommand*\subpoints@nostar[1]{\marginnote{\addpoints{#1}}}

\renewcommand{\arraystretch}{0.8}

\makeatother
%\DebugExSheets{true}    %Asi muestro los ID de las preguntas de este examen
%Paginas sin numeracion (plain para numeracion)
\pagestyle{empty}
%\parindent=0mm
%Encabezados tipo de numeracion etc.
\SetupExSheets{headings=margin-nr}
    \SetupExSheets[points]{name=\itshape\scriptsize Pt.,number-format=\itshape\scriptsize}
\SetupExSheets[points]{name-plural=\itshape\scriptsize Pts.,number-format=\itshape\scriptsize}

\newcommand \cp[1]{%
 \marginnote{\begin{tabular}{|c|}
  \hline\\\\
  \hline
 {\textbf{\points{#1}}}\\
 \hline
\end{tabular}}}




\ExplSyntaxOn
\cs_set_nopar:Npn \exsheets_points_name:n #1
 {
   \bool_if:NT \l__exsheets_points_name_bool
     {
       \,
       \hbox:n
         {
           \bool_if:NTF \l__exsheets_parse_points_bool
             {
               \tl_if_eq:nnTF {#1} { ?? }
                 { \tl_use:N \l__exsheets_points_name_plural_tl }
                 {
                   \fp_compare:nTF { #1 =  1 }
                     { \tl_use:N \l__exsheets_points_name_tl }
                     { \tl_use:N \l__exsheets_points_name_plural_tl }
                 }
             }
             { \tl_use:N \l__exsheets_points_name_plural_tl }
         }
     }
 }
\ExplSyntaxOff

\begin{document}

%una linea
\vspace*{-3cm}\noindent\rule{1,12\textwidth}{0,4pt}
%Tipo examen y curso, Evaluacion
\noindent\makebox[1,12\linewidth]{{\bf{\footnotesize {Control-1/2º Bachillerato S}}}\hfill {\bf{\footnotesize{Primera Evaluación}}}} 
\vspace{-0,35cm}
\noindent\makebox[1,12\linewidth]{{\bf{\tiny{Química Básica.}}}\hfill\ {\bf{\tiny{4 de Noviembre de 2012}}}}\vspace{0,5cm}
\noindent \makebox[1,12\linewidth]{Nombre\dotfill}\vspace{-0,25cm}
\noindent\rule{1,12\textwidth}{0,4pt}
%@ Titre: combustionpropano-1
%@ Domaine: Química Básica
%@ Chapitre: Composición, cantidades en química.
%@ Dificultad: Media-alta
%@ Observaciones: Examen, Estequiometría, rendimiento, combustión
%@ Identificador: 00081

\begin{question}{0.5}
Se hacen reaccionar 20.0 g de propano (\ce{C3H8}) con 50.0 g de \ce{O2}.
Determina el volumen de oxígeno, medido en condiciones normales, que se necesitará para obtener 28.8 g de agua sabiendo que el rendimiento de la reacción es del 80 \%.
\end{question}
\begin{solution}
\end{solution}
\vfill
\hrule
\vspace*{2mm}
{\sc \footnotesize Tabla de puntuaciones\\}{\tiny \begin{center}
\SetupExSheets{
 points/number-format= ,
 counter-format=qu
}
\renewcommand{\arraystretch}{2.2}
\begin{tabular}{|l|*{\numberofquestions}{p{0.8cm}|}c|}\hline
 Pregunta     &
\ForEachQuestion{\centering{\QuestionNumber{#1}}\iflastquestion{}{&}} & Total \\ \hline
 Puntuaci\'on & \
\ForEachQuestion{\centering{\GetQuestionProperty{points}{#1}}\iflastquestion{}{&}} &
\pointssum* \\ \hline
Obtenido  & \ForEachQuestion{\iflastquestion{}{&}} & \\ 
\hline
\end{tabular}

\end{center}
\end{document}